\documentclass{cmspaper}
\usepackage{amssymb}
\usepackage{graphicx}
\usepackage{url}
\usepackage{hyperref}
\usepackage{cite}
\tolerance=1000

\newcommand{\gitbranch}{https://github.com/cms-l1-globaltrigger/mp7_ugt_legacy/blob/cicada_link_11}


\begin{document}
\topmargin = 30mm
\textheight = 225mm

\def\detectornote#1{{Available on CMS information server
\hfill\Large\bf CMS DN #1}
\begin{center}\includegraphics{cms_dn.eps}\end{center}}

\begin{titlepage}

\date{30 March 2023}
\title{Scales for inputs to $\mu$GT
($\varphi$, $\eta$, $p_t$/$E_t$, and others)}

\begin{Authlist}
{H.Bergauer, J.Er\"{o}, M.Jeitler,
%B.Rahbaran,
J.Wittmann, C.-E.Wulz}
\Instfoot{hephy}
{Institute of High Energy Physics of the Austrian Academy of Sciences}

C.Foudas
\Instfoot{Ioannina}
{University of Io\'annina, Greece}

K.Bunkowski, M.Konecki
\Instfoot{Warsaw}
{University of Warsaw, Poland}

L.Uvarov
\Instfoot{INP}
{INP, St. Petersburg, Russia}

C.Battilana, D.Rabady, H.Sakulin
\Instfoot{CERN}
{CERN, Geneva, Switzerland}

J.Brooke
\Instfoot{Bristol}
{University of Bristol, UK}

G.Iles, A.Tapper
\Instfoot{Imperial}
{Imperial College, London, UK}

D.Acosta, I.Furic, A.Madorsky
\Instfoot{Florida}
{University of Florida, Gainesville, USA}

M.Matveev, P.Padley
\Instfoot{Rice}
{Rice University, Houston, USA}

P.Klabbers, W.H.Smith
\Instfoot{Wisconsin}
{Department of Physics, University of Wisconsin, Madison, WI, USA }

\end{Authlist}

\end{titlepage}

With respect to the ``legacy" system, the upgraded GT (uGT or $\mu$GT) has higher requirements concerning precision and amount of data (more input objects of each kind, additional bits for isolation, quality etc.) and also more input bandwidth and computing resources. The additional resources  allow to make the system more uniform and transparent as well as easier to use. The interfaces between uGMT (or $\mu$GMT, replacing GMT) and "Calo Trigger Layer-2" (replacing GCT) have to be defined accordingly.

We are using the new resources as described below (see also proposal~\cite{Darin}; the legacy system's connections are documented in~\cite{muon} for muons and in~\cite{calo} for calo objects) and have introduced the following new features:

1) The hardware  allows for 64 bits per muon object and for 32 bits per any other object (jets, e/$\gamma$, tau, energy sums).

2) All scales are linear (in the legacy system, the muon $p_t$ scale and the calorimeter $\eta$ scale were non-linear).

3) All $\varphi$ scales start at zero (in the legacy system, scales for calo objects started at 350 degrees).

4) Scales are matched to each other so that coarser bins in one system (calo) exactly cover an integer number of smaller bins in another system (muons). The $\varphi$ and $\eta$ scales are as far as possible matched to physical boundaries (tower edges) in the calorimeters.

5) The bin width in  $\varphi$  is 2$\pi$/576 $\sim$ 0.0109... $\sim$ 0.011 for muons and four times wider (2$\pi$/144 $\sim$ 0.0436... $\sim$ 0.044) for all other objects (from calo). These values correspond to 1/8 (for muons) and 1/2 (for calo objects) of a calo tower width in $\varphi$.

The bin width in $\eta$ over the whole $\eta$ range is 1/8 of 0.0870 for muons and 1/2 of 0.0870 for calo objects (0.0870 is the width of a calo tower  in the central rapidity region; at higher pseudorapidity, the physical calo towers get wider). So, for muons the eta bin width is fixed at 0.0870/8=0.010875 while for calo objects it is 0.0870/2=0.0435.

$\eta$ values, which can be positive or negative, are expressed in Two's Complement notation:

So, for muons, which use 9 bits for coding $\eta$, the central value of the bin 0 (-0.010875/2 to +0.010875/2) = 0.0, the left edge of the bins ranges from $-255 \times 0.010875 - 0.010875/2 = -2.7785625$ to $+255 \times 0.010875 - 0.010875/2 = 2.7676875$. The central value of the bins ranges between $\pm 2.773125$ .  The physical $\eta$ range of the muon detectors is about $\pm2.45$, so that not all possible $\eta$ bins are used.

For calo objects, which use 8 bits for coding $\eta$, the left edge of the bins range from $-128 \times 0.0435 = -5.568000$ to $127 \times 0.0435 = 5.524500$ (left edge of the bin 0 = 0.0). The central value of the bins ranges between $\pm 5.546260$. The physical $\eta$ range of the calorimeters is about $\pm5$, so that not all possible $\eta$ bins are used.

\textbf{Remark:} Muon $\eta$ and $\varphi$ raw bits currently not used in uGT. Muon $\varphi$ raw bits [$\varphi$ (out)] are part of the 64 bit muon structure on frames 2 to 6, $\eta$ raw bits are transmitted on frame 1 (see Table~\ref{table:sum_opt_links}).

6) The $p_t$/$E_t$ scale for calo and energy sums objects is identical in step width (0.5 GeV for all systems), starts from 0 (zero) but reaches up to different maximum values for different objects. The highest bin (such as 0x1ff for 9 bits, or 0x7ff for 11 bits, etc.) marks an overflow.\\
The $p_t$ scale for muon objects starts from 0 (zero), but HW index=0 indicates an invalid muon, HW index=1 represents 0 to 0.5 GeV, the step width is 0.5 GeV. The highest bin 0x1ff (for 9 bits) marks an overflow.

7) The new muon structure contains:\\
- "unconstrained $p_t$" scale (8 bits) in steps of 1.0 GeV starting from 0 (zero), HW index=0 indicates an invalid muon, the highest bin 0xff marks an overflow.\\
- "impact parameter" with 2 bits.\\
- "hadronic shower trigger (mus)" bits on bit 31 of MU0, MU2, MU3, MU4 and MU6 objects.

8) The new jet structure contains:\\
- bit 27 will be used to flag a jet as delayed / displaced based on HCAL timing and depth profiles that are indicative of a LLP decay. This bit is referred to as “DISP”. If this bit is set to 1, then the jet has been tagged as an LLP jet.

9) This system allows us to keep a sufficient number of bits for each object free for future use (quality, isolation, possibly tag bits to match uGMT muons to isolation information from the Calorimeter Trigger, etc).

10) For the initial phase, the following numbers of objects are have been implemented: 8 muons, 12 e/$\gamma$'s, 12 taus, 12 jets, and 1 each for the energy sums (ET, ETTEM [ECAL sum - part of the ET data structure], ET$_{miss}$, HT, HT$_{miss}$, ET$_{miss}^{HF}$ and HT$_{miss}^{HF}$).
``Isolated e/$\gamma$'s" do not constitute a separate collection any more but are e/$\gamma$'s marked with two "isolation bit(s)". "Forward jets" also are not in a separate collection any more (their $\eta$ value shows which part of the calorimeter they come from). It is be up to the Calorimeter trigger to rank objects in such a way as to guarantee that not all isolated e/$\gamma$'s will be killed by non-isolated e/$\gamma$'s, or that all central jets will be killed by forward jets.

11) There are ideas to derive electron/gamma signals at high $\eta$ (beyond the range of ECAL) by using the long and short fibers of HF. Therefore, the e/$\gamma$ $\eta$ range has been extended up to $\eta$=5, and the number of e/$\gamma$ objects up to 12. Just as in the case of jets, the Calorimeter trigger will take care that not all central e/$\gamma$'s are killed by such "forward electrons".

12) The minimum bias HF bits are part of the energy sums data structure. Each of the four quantities ET, ET$_{miss}$, HT, HT$_{miss}$ contains HF minimum bias bits on the corresponding MSBs (bits 31..28).

13) The "Towercount" bits (introduced for Heavy-Ion running) are part of the HT data structure (bits 24..12).

14) The "Asymmetry" and "Centrality" bits (introduced for Heavy-Ion running) are part of the ET$_{miss}$, HT$_{miss}$, ET$_{miss}^{HF}$ and HT$_{miss}^{HF}$ data structure.

The following tables (Table~\ref{table:scales_table_1} and ~\ref{table:scales_table_2}) show the bits/resolution per object instance for all objects, including the ones that will be implemented in 2017. "Collection" or "object types" are physical entities such as muons, jets, ET$_{miss}$ etc. "Instances" or "objects" are their individual representatives such as "first muon", "second jet", "third tau" etc.\\
Table~\ref{table:scales_table_3} shows the bits for ZDC objects, see ~\cite{ZDC}.

\begin{table}[ht]
\caption{\bf Scales (muon and calorimeter)}
\vspace{5mm}
\centering
\begin{tabular}{| l | l | l | l | l | l |}
\hline
object	&	collections $\times$ instances	&	parameter	&	range	&	step	&	bits \\
\hline\hline
muon	&	1 * 8	&	$\varphi$ (extrapolated)&	2$\pi$	&	2$\pi$/576$\sim$0.011	&	10 \\
	&		&	$p_t$	&	0..256 GeV 	&	0.5	&	9                    \\
	&		&	quality	&		&		&	4                    \\
	&		&	$\eta$ (extrapolated)	&	-2.45..2.45	&	0.0870/8=0.010875	&	8+1 = 9 \\
	&		&	iso	&		&		&	2                    \\
	&		&	charge sign	&		&		&	1                    \\
	&		&	charge valid	&		&		&	1                    \\
	&		&	index bits	&		&		&	7                    \\
	&		&	$\varphi$ (out) &	2$\pi$	&	2$\pi$/576$\sim$0.011	&	10 \\
	&		&	unconstrained $p_t$	&	0..256 GeV 	&	1.0	&	8                    \\
	&		&	hadronic shower trigger	&		&		&	1                    \\
	&		&	impact parameter	&		&		&	2                    \\
	&		&	TOTAL	&		&		&	64                    \\
% 	&		&		&		&		&	                    \\
\hline
jet	&	1 * 12	&	$E_t$	&	0..1024 GeV	&	0.5	&	11                    \\
	&		&	$\eta$	&	-5..5	&	0.0870/2=0.0435	&	7+1 = 8                    \\
	&		&	$\varphi$	&	2$\pi$	&	2$\pi$/144$\sim$0.044	&	8                    \\
	&		&	DISP	&		&		&	1                    \\
	&		&	quality flags	&		&		&	2                    \\
	&		&	spare	&		&		&	2                    \\
	&		&	TOTAL	&		&		&	32                    \\
% 	&		&		&		&		&	                    \\
\hline
e/$\gamma$	&	1 * 12	&	$E_t$	&	0..256 GeV	&	0.5	&	9                    \\
	&		&	$\eta$	&	-5..5	&	0.0870/2=0.0435	&	7+1 = 8                    \\
	&		&	$\varphi$	&	2$\pi$	&	2$\pi$/144$\sim$0.044	&	8                    \\
	&		&	iso	&		&		&	2                    \\
	&		&	spare	&		&		&	5                    \\
	&		&	TOTAL	&		&		&	32                    \\
% 	&		&		&		&		&	                    \\
\hline
tau	&	1 * 12	&	$E_t$	&	0..256 GeV	&	0.5	&	9                    \\
	&		&	$\eta$	&	-5..5	&	0.0870/2=0.0435	&	7+1 = 8                    \\
	&		&	$\varphi$	&	2$\pi$	&	2$\pi$/144$\sim$0.044	&	8                    \\
	&		&	iso	&		&		&	2                    \\
	&		&	spare	&		&		&	5                    \\
	&		&	TOTAL	&		&		&	32                    \\
% 	&		&		&		&		&	                    \\
\hline
\end{tabular}
\label{table:scales_table_1}
\end{table}

\begin{table}[ht]
\caption{\bf Scales (esums)}
\vspace{5mm}
\centering
\begin{tabular}{| l | l | l | l | l | l |}
\hline
object	&	collections $\times$ instances	&	parameter	&	range	&	step	&	bits \\
\hline\hline
ET	&	1 * 1	&	$E_t$ [ET]	&	0..2048 GeV	&	0.5	&	12                    \\
	&		&	$E_t$ [ETTEM]	&	0..2048 GeV	&	0.5	&	12                    \\
	&		&	spare	&		&		&	4                    \\
	&		&	minimum bias HF	&      0..15	&	n.a.	&	4                    \\
	&		&	TOTAL	&		&		&	32                    \\
% 	&		&		&		&		&	                    \\
\hline
HT	&	1 * 1	&	$E_t$	&	0..2048 GeV	&	0.5	&	12                    \\
	&		&	TOWERCOUNT	&	0..8191	&	1	&	13                    \\
	&		&	spare	&		&		&	3                    \\
	&		&	minimum bias HF	&      0..15	&	n.a.	&	4                    \\
	&		&	TOTAL	&		&		&	32                    \\
% 	&		&		&		&		&	                    \\
\hline
ET$_{miss}$    &   1 * 1   &   $E_t$   &   0..2048 GeV &   0.5 &   12                    \\
    &       &   $\varphi$   &   2$\pi$  &   2$\pi$/144$\sim$0.044   &   8                    \\
    &       &   ASYMET    &  0..255     &    1   &   8                    \\
    &       &   minimum bias HF &      0..15    &   n.a.    &   4                    \\
    &       &   TOTAL   &       &       &   32                    \\
%   &       &       &       &       &                       \\
\hline
HT$_{miss}$    &   1 * 1   &   $E_t$   &   0..2048 GeV &   0.5 &   12                    \\
    &       &   $\varphi$   &   2$\pi$  &   2$\pi$/144$\sim$0.044   &   8                    \\
    &       &   ASYMHT    &  0..255     &    1   &   8                    \\
    &       &   minimum bias HF &      0..15    &   n.a.    &   4                    \\
    &       &   TOTAL   &       &       &   32                    \\
%   &       &       &       &       &                       \\
\hline
ET$_{miss}^{HF}$	&	1 * 1	&   $E_t$   &   0..2048 GeV &   0.5 &   12                    \\
	&		&   $\varphi$   &   2$\pi$  &   2$\pi$/144$\sim$0.044   &   8                    \\
    &       &   ASYMETHF    &  0..255     &    1   &   8                    \\
    &       &   CENT[3:0]    &  4 bits    &       &   4                    \\
	&		&	TOTAL	&		&		&	32                    \\
% 	&		&		&		&		&	                    \\
\hline
HT$_{miss}^{HF}$	&	1 * 1	&   $E_t$   &   0..2048 GeV &   0.5 &   12                    \\
(preliminary	&		&   $\varphi$   &   2$\pi$  &   2$\pi$/144$\sim$0.044   &   8                    \\
definition)	&		&	ASYMHTHF	&	0..255 &	1 &	8                    \\
    &       &   CENT[7:4]    & 4 bits     &       &   4                    \\
	&		&	TOTAL	&		&		&	32                    \\
% 	&		&		&		&		&	                    \\
\hline
\end{tabular}
\label{table:scales_table_2}
\end{table}

\begin{table}[ht]
\caption{\bf Scales (ZDC)}
\vspace{5mm}
\centering
\begin{tabular}{|p{2.0cm}| l | p{2.0cm} | p{2.0cm} | p{2.0cm} | l |}
\hline
object	&	collections $\times$ instances	&	parameter	&	range	&	step	&	bits \\
\hline\hline
ZDC	       &	1 * 2	& EN\_MINUS (ZDC-)	    &	0..1023	   &	1 (?)	  &	10  \\
           &	        & EN\_PLUS (ZDC+)	    &	0..1023	   &	1 (?)	  &	10  \\
\hline
\end{tabular}
\label{table:scales_table_3}
\end{table}

\clearpage

The following pages contain tables for data structure of objects and the data flow of objects on the optical links.

\begin{itemize}
\item A summary of the optical links is shown in Table~\ref{table:sum_opt_links}. (Remark: Muon eta raw bits currently not used in uGT!)
\end{itemize}

\begin{table}[ht]
\caption{\bf Overview optical links (part 1)}
\vspace{5mm}
\scalebox{0.8}{
\centering
\begin{tabular}{|c|c|c|c|c|c|c|c|c|c|c|c|p{1.2cm}|}\hline
% \begin{tabular}{|c|l|l|l|l|l|l|l|l|l|l|p{2.4cm}|}\hline
 & \multicolumn{10}{ c| }{link} \\\hline
frame & 0 & 1 & 2 & 3 & 4 & 5 & 6 & 7 & 8 &\makebox[1.2cm][c]{9} \\\hline\hline
% frame &  & & & & & & & & & & \\\hline\hline
0 & free & free & free & free & EG0 & EG6 & JET0 & JET6 & TAU0 & TAU6\\
 &  &  &  &  &  &  &  &  &  &\\
 &  &  &  &  &  &  &  &  &  &\\
1 & MU0 eta raw & MU2 eta raw & MU4 eta raw & MU6 eta raw & EG1 & EG7  & JET1 & JET7 & TAU1 & TAU7\\
 & on bits 21:13 & on bits 21:13 & on bits 21:13 & on bits 21:13 & & & & & &\\
 & MU1 eta raw & MU3 eta raw & MU5 eta raw & MU7 eta raw & & & & & &\\
 & on bits 30:22 & on bits 30:22 & on bits 30:22 & on bits 30:22 & & & & & &\\
2 & MU0 [31:00]  & MU2 [31:00]  & MU4 [31:00]  & MU6 [31:00]  & EG2 & EG8  & JET2 & JET8  & TAU2 & TAU8\\
 &  &  &  &  &  &  &  &  &  &\\
 &  &  &  &  &  &  &  &  &  &\\
 &  &  &  &  &  &  &  &  &  &\\
3 & MU0 [63:32] & MU2 [63:32] & MU4 [63:32] & MU6 [63:32] & EG3 & EG9  & JET3 & JET9  & TAU3 & TAU9\\
 &  &  &  &  &  &  &  &  &  &\\
 &  &  &  &  &  &  &  &  &  &\\
 &  &  &  &  &  &  &  &  &  &\\
4 & MU1 [31:00]  & MU3 [31:00]  & MU5 [31:00]  & MU7 [31:00]  & EG4 & EG10 & JET4 & JET10 & TAU4 & TAU10\\
 &  &  &  &  &  &  &  &  &  &\\
 &  &  &  &  &  &  &  &  &  &\\
5 & MU1 [63:32] & MU3 [63:32] & MU5 [63:32] & MU7 [63:32] & EG5 & EG11 & JET5 & JET11 & TAU5 & TAU11\\
 &  &  &  &  &  &  &  &  &  &\\
 &  &  &  &  &  &  &  &  &  &\\\hline
\end{tabular}
}
\label{table:sum_opt_links}
\end{table}

\begin{table}[ht]
\caption{\bf Overview optical links (part 2)}
\vspace{5mm}
\scalebox{0.8}{
\centering
\begin{tabular}{|c|c|c|c|c|c|p{1.5cm}|}\hline
 & \multicolumn{6}{ c| }{link} \\\hline
frame & 10 & 11 & 12 & 13 & 14 &\makebox[1.5cm][c]{15} \\\hline\hline
0 & ET, & EN\_MINUS & ExtCond & ExtCond & ExtCond & ExtCond\\
 & ETTEM, & (ZDC-) & [31:0] & [95:64] & [159:128] & [223:192]\\
 & MBT0HFP & [25:16] & & & &\\
1 & HT, & EN\_PLUS & ExtCond & ExtCond & ExtCond & ExtCond\\
 & TOWERCOUNT, & (ZDC+) & [63:32] & [127:96] & [191:160] & [255:224]\\
 & MBT0HFM & [9:0] & & & &\\
2 & ET$_{miss}$, & & free & free & free & free\\
 & ASYMET,& & & & &\\
 & MBT1HFP  & & & & &\\
3 & HT$_{miss}$, & & free & free & free & free\\
 & ASYMHT,& & & & &\\
 & MBT1HFM & & & & &\\
4 & ET$_{miss}^{HF}$ & & free & free & free & free\\
 & ASYMETHF, & & & & &\\
 & CENT[3:0] & & & & &\\
5 & HT$_{miss}^{HF}$ & & free & free & free & free\\
 & ASYMHTHF, & & & & &\\
 & CENT[7:4] & & & & &\\\hline
\end{tabular}
}
\label{table:app:sum_opt_links2}
\end{table}

\begin{itemize}
\item The data structure of a muon object is shown in Table~\ref{table:muon_object}.
\item The definition of the muon $\eta$ scale shown in Table~\ref{table:muon_eta_scale}. The minimum value is -2.45, the maximum +2.45, so the the highest and lowest
bins are "narrower" than other bins.
\item The 7 muon index bits cover a range from 0 to 107 (e.g. index 0-17 is EMTF+ and index 90-107 is EMTF-).
\item The definition of the muon $\varphi$ scale shown in Table~\ref{table:muon_phi_scale}.
\item The definition of the muon quality bits is shown in Table~\ref{table:muon_quality_bits}. It is preliminary, quality ``level x" should be replaced by reliable terms.
\item The definition of the muon isolation bits is shown in Table~\ref{table:muon_iso_bits}. It is preliminary and should be updated when agreed upon.
\item The data flow of muon objects on the optical links is shown in Table~\ref{table:opt_link_muon_obj_0_1}.
\end{itemize}

\begin{table}[ht]
\caption{\bf Data structure of a muon object}
\vspace{5mm}
\centering
\begin{tabular}{|c|c|}\hline
bit(s) & parameter \\\hline\hline
63..62 & impact parameter \\
61 & hadronic shower (mus), on MU0, MU2, MU3, MU4 and MU6 \\
60..53 & unconstrained $p_t$ \\
52..43 & $\varphi$ (out) \\
42..36 & index bits \\
35 & charge valid \\
34 & charge sign \\
33..32 & iso \\
31..23 & $\eta$ (extrapolated) \\
22..19 & quality \\
18..10 & $p_t$ \\
9..0 & $\varphi$ (extrapolated) \\\hline
\end{tabular}
\label{table:muon_object}
\end{table}

\begin{table}[ht]
\caption{\bf $\eta$ scale of muon objects}
\vspace{5mm}
\centering
\begin{tabular}{|c|l|l|c|}\hline
HW index & $\eta$ range &  & $\eta$ bin\\\hline\hline
0x0E1 & 2.4414375 to 2.45 & 224.5$*$0.087/8 to 225.5$*$0.087/8 & 225\\\hline
0x0E0 & 2.4305625 to 2.4414375 & 223.5$*$0.087/8 to 224.5$*$0.087/8 & 224\\\hline
... & ... & ... & ...\\\hline
0x001 & 0.0054375 to 0.0163125 & 0.5$*$0.087/8 to 1.5$*$0.087/8 & 1\\\hline
0x000 & -0.0054375 to 0.0054375 & -0.5$*$0.087/8 to 0.5$*$0.087/8 & 0\\\hline
0x1FF & -0.0163125 to -0.0054375 & -1.5$*$0.087/8 to -0.5$*$0.087/8 & -1\\\hline
0x1FE & -0.0271875 to -0.0054375 & -2.5$*$0.087/8 to -1.5$*$0.087/8 & -2\\\hline
... & ... & ... & ...\\\hline
0x11F & -2.45 to -2.4414375 & -225.5$*$0.087/8 to -224.5$*$0.087/8 & -225\\\hline
\end{tabular}
\label{table:muon_eta_scale}
\end{table}

\begin{table}[ht]
\caption{\bf $\varphi$ scale of muon objects}
\vspace{5mm}
\centering
\begin{tabular}{|c|l|l|c|}\hline
HW index & $\varphi$ range & $\varphi$ range [degrees] & $\varphi$ bin\\\hline\hline
0x000 & 0 to 2$\pi$/576 & 0 to 0.625 & 0\\\hline
0x001 & 2$\pi$/576 to 2$*$2$\pi$/576 & 0.625 to 1.250 & 1\\\hline
... & ... & ... & ...\\\hline
0x23F & 575$*$2$\pi$/576 to 2$\pi$ & 359.375 to 360 & 575\\\hline
\end{tabular}
\label{table:muon_phi_scale}
\end{table}

\begin{table}[ht]
\caption{\bf Definition of muon quality bits}
\vspace{5mm}
\centering
\begin{tabular}{|c|c|}\hline
bits [22..19] & definition \\\hline\hline
0000 & quality "level 0" \\
0001 & quality "level 1" \\
0010 & quality "level 2" \\
0011 & quality "level 3" \\
0100 & quality "level 4" \\
0101 & quality "level 5" \\
0110 & quality "level 6" \\
0111 & quality "level 7" \\
1000 & quality "level 8" \\
1001 & quality "level 9" \\
1010 & quality "level 10" \\
1011 & quality "level 11" \\
1100 & quality "level 12" \\
1101 & quality "level 13" \\
1110 & quality "level 14" \\
1111 & quality "level 15" \\\hline
\end{tabular}
\label{table:muon_quality_bits}
\end{table}

\begin{table}[ht]
\caption{\bf Definition of muon isolation bits}
\vspace{5mm}
\centering
\begin{tabular}{|c|c|}\hline
bits [33..32] & definition \\\hline\hline
00 & not isolated \\
01 & isolated \\
10 & TBD \\
11 & TBD \\\hline
\end{tabular}
\label{table:muon_iso_bits}
\end{table}

\begin{table}[ht]
\caption{\bf Definition of hadronic shower (mus) bits}
\vspace{5mm}
\centering
\begin{tabular}{|c|c|}\hline
muon object & hadronic shower (on bit 61)\\\hline\hline
0 & MUS0 \\
2 & MUS1 \\
3 & MUS2 \\
4 & MUSOOT0 \\
6 & MUSOOT1 \\\hline
\end{tabular}
\label{table:muon_shower_bits}
\end{table}

\begin{table}[ht]
\caption{\bf Definition of muon impact parameter}
\vspace{5mm}
\centering
\begin{tabular}{|c|c|}\hline
bits [63..62] & definition \\\hline\hline
00 & TBD \\
01 & TBD \\
10 & TBD \\
11 & TBD \\\hline
\end{tabular}
\label{table:muon_iso_bits}
\end{table}

\begin{table}[ht]
\caption{\bf Data flow of muon objects 0 and 1 on the optical link \rm(equivalent for objects 2..7)}
\vspace{5mm}
\centering
\begin{tabular}{|c|c|}\hline
frame & objects\\\hline\hline
0 & free \\\hline
1 & free \\\hline
2 & obj. 0, bits 31..0\\\hline
3 & obj. 0, bits 63..32 \\\hline
4 & obj. 1, bits 31..0 \\\hline
5 & obj. 1, bits 63..32 \\\hline
\end{tabular}
\label{table:opt_link_muon_obj_0_1}
\end{table}

\clearpage

\begin{itemize}
\item The data structure of a jet object is shown in Table~\ref{table:jet_object} (bits 31..30 spare bits)
\item The data structure of an e/$\gamma$ object is shown in Table~\ref{table:egamma_object} (bits 31..27 are not defined yet)
\item The data structure of a tau object is shown in Table~\ref{table:tau_object} (bits 31..27 are not defined yet)
\item The definition of isolation bits for e/$\gamma$ and tau is shown in Table~\ref{table:eg_tau_iso_bits}.
\item The definition of the calorimeter $\eta$ scale is shown in Table~\ref{table:calo_eta_scale}. The minimum value is -5.0, the maximum +5.0, so the the highest and lowest
bins are "narrower" other bins.
\item The definition of the calorimeter ET$_{miss}$, ET$_{miss}^{HF}$ and HT$_{miss}$ $\varphi$ scale is shown in Table~\ref{table:calo_phi_scale}.
\end{itemize}

\begin{table}[ht]
\caption{\bf Data structure of a jet object}
\vspace{5mm}
\centering
\begin{tabular}{|c|c|}\hline
bit(s) & parameter \\\hline\hline
31..30 & spare \\
29..28 & quality flags \\
27..27 & DISP \\
26..19 & $\varphi$ \\
18..11 & $\eta$ \\
10..0 & $E_t$ \\\hline
\end{tabular}
\label{table:jet_object}
\end{table}

\begin{table}[ht]
\caption{\bf Data structure of an e/$\gamma$ object}
\vspace{5mm}
\centering
\begin{tabular}{|c|c|}\hline
bit(s) & parameter \\\hline\hline
31..27 & spare \\
26..25 & iso \\
24..17 & $\varphi$ \\
16..9 & $\eta$ \\
8..0 & $E_t$ \\\hline
\end{tabular}
\label{table:egamma_object}
\end{table}

\begin{table}[ht]
\caption{\bf Data structure of a tau object}
\vspace{5mm}
\centering
\begin{tabular}{|c|c|}\hline
bit(s) & parameter \\\hline\hline
31..27 & spare \\
26..25 & iso \\
24..17 & $\varphi$ \\
16..9 & $\eta$ \\
8..0 & $E_t$ \\\hline
\end{tabular}
\label{table:tau_object}
\end{table}

\begin{table}[ht]
\caption{\bf Definition of e/$\gamma$ and tau isolation bits}
\vspace{5mm}
\centering
\begin{tabular}{|c|c|}\hline
bits [26..25] & definition \\\hline\hline
00 & not isolated \\
01 & isolated \\
10 & TBD \\
11 & TBD \\\hline
\end{tabular}
\label{table:eg_tau_iso_bits}
\end{table}

\begin{table}[ht]
\caption{\bf $\eta$ scale of calorimeter objects}
\vspace{5mm}
\centering
\begin{tabular}{|c|l|l|c|}\hline
HW index & $\eta$ range &  & $\eta$ bin \\\hline\hline
0x72 & 4.959 to 5.0025 & 114$*$0.087/2 to 115$*$0.087/2 & 114\\\hline
... & ... & ... & ...\\\hline
0x01 & 0.0435 to 0.087 & 0.087/2 to 2$*$0.087/2 & 1\\\hline
0x00 & 0.0 to 0.0435 & 0 to 0.087/2 & 0\\\hline
0xFF & -0.0435 to 0.0 & -0.087/2 to 0 & -1\\\hline
0xFE & -0.087 to -0.0435  & -2$*$0.087/2 to -0.087/2 & -2\\\hline
... & ... & ... & ...\\\hline
0x8D & -5.0025 to -4.959 & -115$*$0.087/2 to -114$*$0.087/2 & -115\\\hline
\end{tabular}
\label{table:calo_eta_scale}
\end{table}

\begin{table}[ht]
\caption{\bf $\varphi$ scale of calorimeter objects, ET$_{miss}$, ET$_{miss}^{HF}$, HT$_{miss}$ (and HT$_{miss}^{HF}$ [preliminary definition])}
\vspace{5mm}
\centering
\begin{tabular}{|c|l|l|c|}\hline
HW index & $\varphi$ range & $\varphi$ range [degrees] & $\varphi$ bin\\\hline\hline
0x00 & 0 to 2$\pi$/144 & 0 to 2.5 & 0\\\hline
0x01 & 2$\pi$/144 to 2$*$2$\pi$/144 & 2.5 to 5.0 & 1\\\hline
... & ... & ... & ...\\\hline
0x8F & 143$*$2$\pi$/144 to 2$\pi$ & 357.5 to 360 & 143\\\hline
\end{tabular}
\label{table:calo_phi_scale}
\end{table}

\clearpage

\begin{itemize}
\item The data flow of e/$\gamma$, tau and jet objects 0..5 on an optical link is shown in Table~\ref{table:opt_link_egamma_obj_0_5}.
\item The data flow of e/$\gamma$, tau and jet objects 6..11 on an optical link is shown in Table~\ref{table:opt_link_egamma_obj_6_11}.
\end{itemize}

\begin{table}[ht]
\caption{\bf Data flow of e/$\gamma$, tau and jet objects 0..5 on optical link}
\vspace{5mm}
\centering
\begin{tabular}{|c|c|}\hline
frame & objects\\\hline\hline
0 & obj. 0 \\\hline
1 & obj. 1 \\\hline
2 & obj. 2 \\\hline
3 & obj. 3 \\\hline
4 & obj. 4 \\\hline
5 & obj. 5 \\\hline
\end{tabular}
\label{table:opt_link_egamma_obj_0_5}
\end{table}

\begin{table}[ht]
\caption{\bf Data flow of e/$\gamma$, tau and jet objects 6..11 on optical link}
\vspace{5mm}
\centering
\begin{tabular}{|c|c|}\hline
frame & objects\\\hline\hline
0 & obj. 6 \\\hline
1 & obj. 7 \\\hline
2 & obj. 8 \\\hline
3 & obj. 9 \\\hline
4 & obj. 10 \\\hline
5 & obj. 11 \\\hline
\end{tabular}
\label{table:opt_link_egamma_obj_6_11}
\end{table}

\clearpage

\begin{itemize}
\item The data flow of energy sums on the optical link is shown in Table~\ref{table:opt_link_esums}.
\item The data structure of ET (including ETTEM and MBT0HFP), HT (including TOWERCOUNT and MBT0HFM), ET$_{miss}$ (including MBT1HFP), HT$_{miss}$ (including MBT1HFM), ET$_{miss}^{HF}$    ET$_{miss}^{HF}$
and HT$_{miss}^{HF}$
is shown in Tables~\ref{table:ett_object}, ~\ref{table:ht_object}, ~\ref{table:etm_object}, ~\ref{table:htm_object}, ~\ref{table:etmhf_object} and ~\ref{table:htmhf_object}.
\item The definition of minimum bias HF, ETTEM, TOWERCOUNT, Asymmetry and Centrality bits is shown in ~\ref{table:min_bias_def}, ~\ref{table:ettem_def}, ~\ref{table:tower_count_def}
~\ref{table:asymmetry_def} and ~\ref{table:centrality_def}.
\item The data structure of ZDC is shown in Table~\ref{table:zdc_struct}
\end{itemize}

\begin{table}[ht]
\caption{\bf Data flow of energy sums on optical link}
\vspace{5mm}
\centering
\begin{tabular}{|c|c|}\hline
frame & objects \\\hline\hline
0 & ET, ETTEM, MBT0HFP \\\hline
1 & HT, TOWERCOUNT, MBT0HFM \\\hline
2 & ET$_{miss}$, ASYMET, MBT1HFP \\\hline
3 & HT$_{miss}$, ASYMHT, MBT1HFM \\\hline
4 & ET$_{miss}^{HF}$, ASYMETHF, CENT[3:0] \\\hline
5 & HT$_{miss}^{HF}$, ASYMHTHF, CENT[7:4] \\\hline
\end{tabular}
\label{table:opt_link_esums}
\end{table}

\begin{table}[ht]
\caption{\bf Data structure of ET \rm(including ETTEM and MBT0HFP)}
\vspace{5mm}
\centering
\begin{tabular}{|c|c|}\hline
bit(s) & parameter \\\hline\hline
31..28 & minimum bias HF+ threshold 0 \\
27..24 & spare \\
23..12 & $E_t$ [ETTEM] \\
11..0 & $E_t$ [ET] \\\hline
\end{tabular}
\label{table:ett_object}
\end{table}

\begin{table}[ht]
\caption{\bf Data structure of HT \rm(including TOWERCOUNT and MBT0HFM)}
\vspace{5mm}
\centering
\begin{tabular}{|c|c|}\hline
bit(s) & parameter \\\hline\hline
31..28 & minimum bias HF- threshold 0 \\
27..25 & spare \\
24..12 & TOWERCOUNT \\
11..0 & $E_t$ \\\hline
\end{tabular}
\label{table:ht_object}
\end{table}

\begin{table}[ht]
\caption{\bf Data structure ET$_{miss}$ \rm(including MBT1HFP)}
\vspace{5mm}
\centering
\begin{tabular}{|c|c|}\hline
bit(s) & parameter \\\hline\hline
31..28 & minimum bias HF+ threshold 1 \\
27..20 & ASYMET \\
19..12 & $\varphi$ \\
11..0 & $E_t$ \\\hline
\end{tabular}
\label{table:etm_object}
\end{table}

\begin{table}[ht]
\caption{\bf Data structure HT$_{miss}$ \rm(including MBT1HFM)}
\vspace{5mm}
\centering
\begin{tabular}{|c|c|}\hline
bit(s) & parameter \\\hline\hline
31..28 & minimum bias HF- threshold 1 \\
27..20 & ASYMHT \\
19..12 & $\varphi$ \\
11..0 & $E_t$ \\\hline
\end{tabular}
\label{table:htm_object}
\end{table}

\begin{table}[ht]
\caption{\bf Data structure ET$_{miss}^{HF}$}
\vspace{5mm}
\centering
\begin{tabular}{|c|c|}\hline
bit(s) & parameter \\\hline\hline
31..28 & CENT[3:0] \\
27..20 & ASYMETHF \\
19..12 & $\varphi$ \\
11..0 & $E_t$ \\\hline
\end{tabular}
\label{table:etmhf_object}
\end{table}

\begin{table}[ht]
\caption{\bf Data structure HT$_{miss}^{HF}$}
\vspace{5mm}
\centering
\begin{tabular}{|c|c|}\hline
bit(s) & parameter \\\hline\hline
31..28 & CENT[7:4] \\
27..20 & ASYMHTHF \\
19..12 & $\varphi$ \\
11..0 & $E_t$ \\\hline
\end{tabular}
\label{table:htmhf_object}
\end{table}

\begin{table}[ht]
\caption{\bf ECAL sum definition (ETTEM) \rm(in energy sums structure)}
\vspace{5mm}
\centering
\begin{tabular}{|c|c|c|c|c|}\hline
objects & acronym & frame & object & bits \\\hline\hline
ECAL sum & ETTEM & 0 & ET & 23..12 \\\hline
\end{tabular}
\label{table:ettem_def}
\end{table}

\begin{table}[ht]
\caption{\bf Definition of ``Towercount"   \rm(in energy sums structure; introduced for Heavy-Ion running)}
\vspace{5mm}
\centering
\begin{tabular}{|c|c|c|c|c|}\hline
objects & acronym & frame & object & bits \\\hline\hline
Towercount & TOWERCOUNT & 1 & HT & 24..12 \\\hline
\end{tabular}
\label{table:tower_count_def}
\end{table}

\begin{table}[ht]
\caption{\bf Minimum bias HF definition \rm(in energy sums structure)}
\vspace{5mm}
\centering
\begin{tabular}{|c|c|c|c|c|}\hline
objects & acronym & frame & objects & bits \\\hline\hline
minimum bias HF+ threshold 0 & MBT0HFP & 0 & ET & 31..28 \\
minimum bias HF- threshold 0 & MBT0HFM & 1 & HT & 31..28 \\
minimum bias HF+ threshold 1 & MBT1HFP & 2 & ET$_{miss}$ & 31..28 \\
minimum bias HF- threshold 1 & MBT1HFM & 3 & HT$_{miss}$ & 31..28 \\\hline
\end{tabular}
\label{table:min_bias_def}
\end{table}

\begin{table}[ht]
\caption{\bf "Asymmetry" definition \rm(in energy sums structure)}
\vspace{5mm}
\centering
\begin{tabular}{|c|c|c|c|c|}\hline
objects & acronym & frame & objects & bits \\\hline\hline
Asymmetry of ET & ASYMET & 2 & ET$_{miss}$ & 27..20 \\
Asymmetry of HT & ASYMHT & 3 & HT$_{miss}$ & 27..20 \\
Asymmetry of ETHF & ASYMETHF & 4 & ET$_{miss}^{HF}$ & 27..20 \\
Asymmetry of HTHF & ASYMHTHF & 5 & HT$_{miss}^{HF}$ & 27..20 \\\hline
\end{tabular}
\label{table:asymmetry_def}
\end{table}

\begin{table}[ht]
\caption{\bf "Centrality" definition \rm(in energy sums structure)}
\vspace{5mm}
\centering
\begin{tabular}{|c|c|c|c|c|}\hline
objects & acronym & frame & objects & bits \\\hline\hline
Centrality bits [3:0] & CENT[3:0] & 4 & ET$_{miss}^{HF}$ & 31..28 \\
Centrality bits [7:4] & CENT[7:4] & 5 & HT$_{miss}^{HF}$ & 31..28 \\\hline
\end{tabular}
\label{table:centrality_def}
\end{table}

\begin{table}[ht]
\caption{\bf Data structure ZDC}
\vspace{5mm}
\centering
\begin{tabular}{|c|c|c|}\hline
frame & bit(s) & parameter \\\hline\hline
0 & 25..16 & EN\_MINUS (ZDC-)\\
1 & 9..0 & EN\_PLUS (ZDC+)\\\hline
\end{tabular}
\label{table:zdc_struct}
\end{table}

\clearpage

\begin{thebibliography}{00}

\bibitem {Darin} 
\url{https://indico.cern.ch/getFile.py/access?contribId=4\&sessionId=0\&resId=0\&materialId=slides\&confId=206223}

\bibitem {muon}
\url{https://cds.cern.ch/record/1364673/files/in04_006.pdf}

\bibitem {calo}
\url{https://cds.cern.ch/record/1364214/files/in02_069.pdf}

\bibitem {ZDC}
\href{\gitbranch/doc/scales_inputs_2_ugt/ZDC_uhtr_spec.pdf}{\texttt{Description of ZDC structure (ZDC\_uhtr\_spec.pdf)}}

\end{thebibliography}

\end{document}

